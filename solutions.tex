\documentclass[12pt]{amsart}
\usepackage{amsmath}
\usepackage{amsthm}
\usepackage{amsfonts}
\usepackage{amssymb}
\usepackage[margin=1in]{geometry}
\usepackage{hyperref}
\hypersetup{
    colorlinks=true,
    linkcolor=blue
}

\theoremstyle{definition}
\newtheorem{theorem}{Theorem}[section]
\newtheorem{lemma}[theorem]{Lemma}
\newtheorem{definition}[theorem]{Definition}
\newtheorem{corollary}[theorem]{Corollary}
\newtheorem{proposition}[theorem]{Proposition}
\newtheorem{conjecture}[theorem]{Conjecture}
\newtheorem{remark}[theorem]{Remark}
\newtheorem{example}[theorem]{Example}
\newtheorem{problem}[theorem]{Problem}
\newtheorem{notation}[theorem]{Notation}
\newtheorem{question}[theorem]{Question}
\newtheorem{caution}[theorem]{Caution}

\begin{document}

\title{Homework 1}

\maketitle

For this week, please answer the following questions from the text. 
I've copied the problem itself below and the question numbers for 
your convenience. 

\begin{enumerate}
	\item (1.2) Decrypt each of the following Caesar encryptions by trying the various 
		possible shifts until you obtain readable text.
		\begin{itemize}
			\item \texttt{LWKLQNWKDWLVKDOOQHYHUVHHDELOOERDUGORYHOBDVDWUHH}
			\item \texttt{UXENRBWXCUXENFQRLQJUCNABFQNWRCJUCNAJCRXWORWMB}
			\item \texttt{BGUTBMBGZTFHNLXMKTIPBMAVAXXLXTEPTRLEXTOXKHHFYHKMAXFHNLX}
		\end{itemize}

        Decrypted text
        \begin{itemize}
            \item I think that I shall never see a billboard lovely as a tree.
            \item Love is not love which alters when it alteration finds.
            \item In baiting a mouse trap with cheese always leave room for the mouse.
        \end{itemize}
	\item (1.3) Use the simple substitution table below
	\begin{center}
		\begin{tabular}{|c |c |c |c |c |c |c |c |c |c |c |c |c |c |c |c |c |c |c |c |c |c |c |c| c| c|}
			\hline
			a & b & c & d & e & f & g & h & i & j & k & l & m & n & o & 
			p & q & r & s & t & u & v & w & x & y & z \\
			\hline
			S & C & J & A & X & U & F & B & Q & K & T & P & R & W & E & 
			Z & H & V & L & I & G & Y & D & N & M & O \\
			\hline
		\end{tabular}
	\end{center}
	\begin{enumerate}
		\item Encrypt the plaintext message
		\begin{center}
			\texttt{The gold is hidden in the garden.}
		\end{center}
		\item Make a decryption table, that is, make a table in which the ciphertext 
			alphabet is in order from A to Z and the plaintext alphabet is mixed up.
		\item Use your decryption table from (b) to decrypt the following message.
		\begin{center}
			\texttt{IBXLX JVXIZ SLLDE VAQLL DEVAU QLB}
		\end{center}
	\end{enumerate}
        Solutions
        \begin{enumerate}
            \item IBX FEPA QL BQAAXW QW IBX FSVAXW
            \item Decryption table
                \begin{center}
                    \begin{tabular}{|c |c |c |c |c |c |c |c |c |c |c |c |c |c |c |c |c |c |c |c |c |c |c |c| c| c|}
        			\hline
        			A & B & C & D & E & F & G & H & I & J & K & L & M & N & O & 
        			P & Q & R & S & T & U & V & W & X & Y & Z \\
        			\hline
        			d & h & b & w & o & g & u & q & t & c & j & s & y & x & z & 
        			l & i & m & a & k & f & r & n & e & v & p \\
        			\hline
        		\end{tabular}
                \end{center}
            \item the secret password is swordfish
        \end{enumerate}
\item (1.4.c) Each of the following messages has been encrypted using a simple
	substitution cipher. Decrypt them. For your convenience, we have given
	you a frequency table and a list of the most common bigrams that appear
	in the ciphertext. (If you do not want to recopy the ciphertexts by
	hand, they can be downloaded or printed from the web site listed in the
	preface.) In order to make this one a bit more challenging, we have
	removed all occurrences of the word “the” from the plaintext. 

	“A Brilliant Detective”
	\begin{center}
		\ttfamily
		GSZES GNUBE SZGUG SNKGX CSUUE QNZOQ EOVJN VXKNG XGAHS AWSZZ
		BOVUE SIXCQ NQESX NGEUG AHZQA QHNSP CIPQA OIDLV JXGAK CGJCG
		SASUB FVQAV CIAWN VWOVP SNSXV JGPCV NODIX GJQAE VOOXC SXXCG
		OGOVA XGNVU BAVKX QZVQD LVJXQ EXCQO VKCQG AMVAX VWXCG OOBOX
		VZCSO SPPSN VAXUB DVVAX QJQAJ VSUXC SXXCV OVJCS NSJXV NOJQA
		MVBSZ VOOSH VSAWX QHGMV GWVSX CSXXC VBSNV ZVNVN SAWQZ ORVXJ
		CVOQE JCGUW NVA
	\end{center}
		
	The ciphertext contains $313$ letters. Here is a frequency table: 
	\begin{center}
		\ttfamily
		\begin{tabular}{|c||c |c |c |c |c |c |c |c |c |c |c |c |c |c |c |c |c |c |c |c |c |c |c| c| c|}
			\hline
			& V & S & X & G & A & O & Q & C & N & J & U & Z & E & W
			& B & P & I & H & K & D & M & L & R & F \\
			\hline
			Freq  & 39 & 29 & 29 & 22 & 21 & 21 & 20 & 20 & 19 & 13
			      & 11 & 11 & 10 & 8 & 8 & 6 & 5 & 5 & 5 & 4 & 3 &
			2 & 1 & 1 \\
			\hline
		\end{tabular}
	\end{center}

	The most frequent bigrams are: \texttt{XC} (10 times), \texttt{NV} (7
	times), and \texttt{CS}, \texttt{OV}, \texttt{QA}, and \texttt{SX} (6
	times each).	
        Decrypted text:
        \begin{center}
            Decrypted: I am fairly familiar with all forms of secret writing and am myself author of a trifling monograph upon subject in which i analyze one hundred separate ciphers but i confess that this is entirely new to me object of those who invented this system has apparently been to conceal that these characters convey a message and to give idea that they are mere random sketches of children
        \end{center}
        To decrypt this code, I first looked at the frequency tables for the individual digits and the bigrams. I was able to immediately determine that ‘V’ corresponded to ‘e’. I then looked at the common bigrams and noted which letters were repeated. I then looked at the frequency of digits within bigrams to make a couple of educated guesses. Afterwards, I looked up common trigrams as well as letters that appear most often in pairs. This is how I discovered the identities of ‘s’ and ‘l’. In addition, I knew that my predictions were correct when, early on, I had solved for ‘hundred seParate’. I was then able to infer the rest of the letters from context clues.

		
	\item (1.5) Suppose that you have an alphabet of 26 letters.
		\begin{enumerate}
			\item  How many possible simple substitution ciphers are there?
			\item  A letter in the alphabet is said to be fixed if the encryption of the letter is the
				letter itself. How many simple substitution ciphers are there that leave:
			\begin{enumerate}
				\item No letters fixed?
				\item At least one letter fixed?
				\item Exactly one letter fixed?
				\item At least two letters fixed?
			\end{enumerate}
			(Part (b) is quite challenging! You might try doing the problem first with an alphabet 
			of four or five letters to get an idea of what is going on.)
		\end{enumerate}
            Problem 4 Solutions
            \begin{enumerate}
                \item This is equivalent to the number of permutations of the alphabet, which is $26! = 403291461126605635584000000$
                \item
                \begin{enumerate}
                    \item This can be counted using the principle of inclusion-exclusion. We begin with $26!$ total simple substitution ciphers. We then subtract all of the ciphers with one fixed element, which is ${26\choose1} * 25!$, as we are choosing 1 of 26 letters to fix and permuting the others in any other way. However, this subtracted the case where 2 letters are fixed, so this is added back as ${26 \choose 2} * 24!$. This pattern of alternating addition and subtraction continues with terms matching the form
                    \begin{center}
                        ${26 \choose i} * (26 - i)!$.
                    \end{center}
                    which by the definition of $n \choose k$ is equal to
                    \begin{center}
                        ${\frac{26!}{k! (26 - k)!} * (26 - k)!} = \frac{26!}{k!}$
                    \end{center}
                    This simplifies our expression to be
                    \begin{center}
                        $26! - \frac{26!}{1!} + \frac{26!}{2!} - \dots + \frac{26!}{26!}$
                    \end{center}
                    which is equivalent to the sum
                    \begin{center}
                         \[ 26! \sum_{k=0}^{26} \frac{(-1)^k}{k!} \approx 1.48362637\times10^{26} \]
                    \end{center}

                    \item All possible simple substitution ciphers can be broken into either having no fixed letters or having at least one fixed letter. We can then subtract the number without any fixed letters from the total:
                    \begin{center}
                        \[26! - ( 26! \sum_{k=0}^{26} \frac{(-1)^k}{k!} ) \approx 2.54928824\times10^{26}\]
                    \end{center}

                    \item We begin by choosing a letter to fix, which is equal to $26 \choose 1 = 26$. We can then multiply this by the number of ciphers on 25 letters without any fixed letters, which is \[ 25! \sum_{k=0}^{25} \frac{(-1)^k}{k!} \], so the total with one fixed is 
                    \begin{center}
                        \[ 26! * (25! \sum_{k=0}^{25} \frac{(-1)^k}{k!}) \]
                    \end{center}

                    \item The number of ciphers with at least two letters fixed is the number with at least one fixed minus the number with exactly one fixed:
                    \begin{center}
                        \[26! - ( 26! \sum_{k=0}^{26} \frac{(-1)^k}{k!} ) - ( 26! * (25! \sum_{k=0}^{25} \frac{(-1)^k}{k!}) )\]
                    \end{center}
                \end{enumerate}
            \end{enumerate}
            
\end{enumerate}

\end{document}
